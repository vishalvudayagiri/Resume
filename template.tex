%%%%%%%%%%%%%%%%%%%%%%%%%%%%%%%%%%%%%%%%%
% Twenty Seconds Resume/CV
% LaTeX Template
% Version 1.1 (8/1/17)
%
% This template has been downloaded from:
% http://www.LaTeXTemplates.com
%
% Original author:
% Carmine Spagnuolo (cspagnuolo@unisa.it) with major modifications by 
% Vel (vel@LaTeXTemplates.com)
%
% License:
% The MIT License (see included LICENSE file)
%
%%%%%%%%%%%%%%%%%%%%%%%%%%%%%%%%%%%%%%%%%

%----------------------------------------------------------------------------------------
%	PACKAGES AND OTHER DOCUMENT CONFIGURATIONS
%----------------------------------------------------------------------------------------

\documentclass[letterpaper]{twentysecondcv} % a4paper for A4

%----------------------------------------------------------------------------------------
%	 PERSONAL INFORMATION
%----------------------------------------------------------------------------------------

% If you don't need one or more of the below, just remove the content leaving the command, e.g. \cvnumberphone{}

\profilepic{} % Profile picture                  

\cvname{Vishal \\ Ramakrishna} % Your name
%\cvjobtitle{K Nair} % Job title/career 

\cvdate{} % Date of birth
\cvaddress{vishalvudayagiri@gmail.com} % Email address
\cvnumberphone{+49 17656960190} % Phone number
\cvsite{Duisburg, Germany} % Personal website
\cvmail{linkedin/vishalvudayagiri} % Email address
%\cvaddress{dipinknair619@gmail.com} % Email address
%\cvmailto{dipinknair619@gmail.com} % Email address

%----------------------------------------------------------------------------------------

\begin{document}


%----------------------------------------------------------------------------------------
%	 Education
%----------------------------------------------------------------------------------------

\Education{M.Sc. Information and Communication Engineering\\ Technical University of Darmstadt\\ 2018 | GPA:2.54/5.0 (80-90 \%)
\newline \newline B.E. Electronics and Communication Engineering\\ Visveswaraya Technological Institute\\ 2014 | GPA:8.92/10 (81.7 \%)
\newline \newline Class XII \newline SBMJC V.V.Puram, Bangalore\\ 2010 | 94 \%} % To have no Education section, just remove all the text and leave \Education{}

%----------------------------------------------------------------------------------------
%	 SKILLS
%----------------------------------------------------------------------------------------

% Skill bar section, each skill must have a value between 0 an 6 (float)
\skills{Programming : Assembly, C, Python, Unix Scripting, C++, Basics of Verilog
\newline
\newline  Tools : MATLAB, Agilent ADS, CST Microwave Studio, MS-Office, Arduino
\newline
\newline Hardware : TCI6638K2K Multicore DSP, LA9358, LA1224 NXP SoC, Arduino
\newline
\newline  Languages: English, German, Kannada, Hindi}

\Strength{}

\Hobbies{}


%%----------------------------------------------------------------------------------------
%%	 SKILLS
%%----------------------------------------------------------------------------------------
%
%% Skill bar section, each skill must have a value between 0 an 6 (float)
%\skills{{pursuer of rabbits/5.8},{good manners/4},{outgoing/4.3},{polite/4},{Java/0.01}}
%
%%------------------------------------------------
%
%% Skill text section, each skill must have a value between 0 an 6
%\skillstext{{lovely/4},{narcissistic/3}}

%----------------------------------------------------------------------------------------

\makeprofile % Print the sidebar


%----------------------------------------------------------------------------------------
%	 EXPERIENCE
%----------------------------------------------------------------------------------------

%\section{Work Experience}
%
%\begin{twenty} % Environment for a list with descriptions
%	\twentyitem{2016 Sep}{Senior FEA Analyst - General Motors | Bangalore}{2.5 Years}{- Gear Macro \& Micro geometry optimisaiton and stress analysis\\- Component level FE analysis of shafts, Differential, Gears (Manual \&   Automatic)\\- Subroutine development of bearing in Abaqus\\- Meshing automation \\- Participated on teardown of transmissions and engines\\- Various postprocessing automation using Excel VBA\\- DFSS Green Belt}
%	\twentyitem{May-Aug'16}{Student Intern - General Motors | Bangalore}{3 Months}{- Automation of Abaqus system modeling(substrucure method) using   VBA scripting\\- Correlation of MASTA and Abaqus on simple shaft model for validating substructure method}
%	\twentyitem{May-Aug'15}{Student Intern - Smartron India Pvt Ltd | Bangalore.}{3 months}{Industrial designing of smartphone}
%	%\twentyitem{<dates>}{<title>}{<location>}{<description>}
%\end{twenty}

\section{Work Experience}

\begin{twenty} % Environment for a list with descriptions
	\twentyitem{2018 May - }{DSP Engineer - CommAgility | Duisburg, Germany}{1.7 Years}{- Embedded Software Developer PHY L1 on baseband Digital Signal Processors for small cell 5G NR gNB, LTE/LTE-A/private LTE eNodeB.\\- Implemented UL chain symbol level at gNB (3GPP Rel.15) including physical channels such as PUCCH, PUSCH, SRS, on platforms such as TCI6638K2K Multicore DSP + ARM Keystone II SoC, and Vector signal processors NXP.\\- Implemented primitive kernel functions in assembly optimized for FIR filters, FFT, Channel estimation, MMSE Equalizer.\\- Implemented primitive modules in the host side to read/write shared memory over TCP/IP, schedule tasks, memory management etc.\\- Support to generation of Test Vectors, verifying with VSA, and automating the testing using bash/python scripts.\\- Simulations in MATLAB for 5G NR Tx/Rx Beamforming, Channel Estimation, Antenna modelling at gNB.}
	\twentyitem{Aug-Feb'18}{Master Thesis - TU Darmstadt | Darmstadt, Germany}{10 months}{- Channel estimation for Massive MIMO systems based on compressive sensing principles using receive power measurement.\\- Simulated an UL Massive MIMO system with Receive Beamforming and perform Channel Estimation using a novel algorithm based on Compressed sensing principles (STELA).}
	\twentyitem{Aug-Feb'17}{Student Intern - Intel | Munich, Germany}{6 Months}{- System level plan and signal generation in MATLAB for RF receiver test cases for addition of CDMA2000 RAT in Intel Modems (3GPP2).\\- DL receiver chain modelling in MATLAB and preparation of reference at UE side.}
	\twentyitem{Apr-Jun'16}{Student Intern - Rohde \& Schwarz GmBh | Munich}{3 months}{- Channel sounding measurement campaigns for channel parameter estimation at mm-wave frequencies.\\- Experience to work with R~\&~S signal generators and frequency analyzers.}
	%\twentyitem{<dates>}{<title>}{<location>}{<description>}
\end{twenty}
%%----------------------------------------------------------------------------------------
%%	 INTERESTS
%%----------------------------------------------------------------------------------------
%
%\section{Interests}
%
%Manufacturing, Design \& FEA Analysis
%\newline Robbotics, Data Science \& Automation
%

%----------------------------------------------------------------------------------------
%	 PUBLICATIONS
%----------------------------------------------------------------------------------------

\section{Academic Projects}

\begin{twenty} % Environment for a list with descriptions
	\twentyitem{Apr-Aug'15}{Search free based direction of arrival estimation algorithms}{}{- Simulated signal processing algorithms MUSIC, ESPRIT, Root-MUSIC, Root-RARE for fully and partly calibrated array of antenna systems.}
	\twentyitem{Oct-Mar'16}{BAW based RF duplexer module}{}{- Designed and fabricated a RF duplexer module using Agilent ADS with Power Amplifier and BAW filter at GSM bands.\\- S-parameters of the fabricated device were measured to compare with the simulations of the design.}
	\twentyitem{Oct-Mar'16}{Design, implementation and synthesis of a MIPS-I processor core}{}{- Implemented architecture was a 4 stage pipeline with a load store 32 bit instruction set and synthesized on a FPGA.}
	\twentyitem{Oct-Mar'16}{Optimal luminance control of networked lights}{}{- Programmed an Arduino Yun using Python to automatically control luminance of the networked lights.}
	\twentyitem{Sep-Aug'14}{Bachelor Thesis}{}{- Eyes free interaction for mobile reading devices.\\- Constructed a mobile prototyping device based on the Arduino Yun capable of capturing an image document, comprehending the text and converting text to speech}
	%\twentyitem{<dates>}{<title>}{<location>}{<description>}
\end{twenty}

%----------------------------------------------------------------------------------------
%	 AWARDS
%----------------------------------------------------------------------------------------
%\section{Achievements}

%- Secured 2763 in JEE 2011 out of 0.5 Million students appeared\\- Secured State level 101 AIEEE  out of 1.5 Million students appeared\\- Secured 56th rank in KEAM 2011 out of 0.135 Million students appeared\\- Represented IIT Hyderabad in ABU Robocon - 2014 held in Pune


%\section{Achievements}
%
%\begin{twentyshort} % Environment for a short list with no descriptions
%	\twentyitemshort{2011}{Secured 2763 in JEE out of 0.5 Million students appeared\\Secured State level 101 AIEEE  out of 1.5 Million students appeared\\Secured 56th rank in KEAM out of 0.135 Million students appeared}
%	%\twentyitemshort{<dates>}{<title/description>}
%\end{twentyshort}
%Studied nonlocal theory of elasticity which takes account of remote action forces between atoms
%
%This causes the stresses to depend on the strains not only at an individual point
%under consideration, but at all points of the body
%
%Mechanical behavior calculated by Classical elastic theory will differ from non
%local theory



%
%\begin{twenty} % Environment for a short list with no descriptions
%	\twentyitem{Honors}{Nonlocal Effect on the Frequency Analysis of Carbon Nanotubes}{\\ Dr. Ashok Kumar Pandey}{}
%
%	\twentyitem{Academic}{Matlab code for the area filling algorithm for 3d printing}{\\ Dr. Surya. S Kumar}{}
%
%	\twentyitem{Academic}{Epoxy photo elasticity experiment to find stress concentration factor}{\\ Dr. M. Ramji}{}
%
%	\twentyitem{Academic}{Stress and strain analysis of Ice Screw using FEA}{\\  Dr. Viswanath Chinthapenta}{}

%	\twentyitem{Mini}{micro cantilever beam for sensing application}{\\ Dr.Prem Pal}{Making a model to explain charecterestics and working of cantilever beam for practical aspects}


%\twentyitem{Indpendent}{Produced magnetic breaking(without any friction) on Aluminium wheel with help of strong magnet}{}{}
%
%
%\end{twenty}
%----------------------------------------------------------------------------------------
%	 OTHER INFORMATION
%----------------------------------------------------------------------------------------

%\section{Other information}
%
%\subsection{Review}



%----------------------------------------------------------------------------------------
%	 SECOND PAGE EXAMPLE
%----------------------------------------------------------------------------------------

%\newpage % Start a new page

%\makeprofile % Print the sidebar

%\section{Other information}

%\subsection{Review}

%Alice approaches Wonderland as an anthropologist, but maintains a strong sense of noblesse oblige that comes with her class status. She has confidence in her social position, education, and the Victorian virtue of good manners. Alice has a feeling of entitlement, particularly when comparing herself to Mabel, whom she declares has a ``poky little house," and no toys. Additionally, she flaunts her limited information base with anyone who will listen and becomes increasingly obsessed with the importance of good manners as she deals with the rude creatures of Wonderland. Alice maintains a superior attitude and behaves with solicitous indulgence toward those she believes are less privileged.

%\section{Other information}

%\subsection{Review}

%Alice approaches Wonderland as an anthropologist, but maintains a strong sense of noblesse oblige that comes with her class status. She has confidence in her social position, education, and the Victorian virtue of good manners. Alice has a feeling of entitlement, particularly when comparing herself to Mabel, whom she declares has a ``poky little house," and no toys. Additionally, she flaunts her limited information base with anyone who will listen and becomes increasingly obsessed with the importance of good manners as she deals with the rude creatures of Wonderland. Alice maintains a superior attitude and behaves with solicitous indulgence toward those she believes are less privileged.

%----------------------------------------------------------------------------------------

\end{document} 
